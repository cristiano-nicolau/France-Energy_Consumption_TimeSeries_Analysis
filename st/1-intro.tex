\section{Introduction}

Electricity consumption is a crucial indicator of a country's economic and social well-being, reflecting not only industrial and commercial activity but also the population's living standards. Understanding and forecasting it are fundamental for energy planning, infrastructure management, and the development of sustainability policies. In a global scenario of growing concern over climate change and energy transition, the ability to accurately predict electricity demand becomes even more relevant. Accurate forecasts allow for optimizing energy generation and distribution, preventing grid overloads, and anticipating future needs, contributing to the security and efficiency of electrical systems.\\

This study is dedicated to the analysis and forecasting of monthly electricity consumption in France, a country with one of Europe's largest economies and a complex and diverse energy sector, characterized by a strong reliance on nuclear power (84.7\%) and increasing investment in renewable energies \cite{EDF2012}. To this end, we will use a time series, compiled by Eurostat, covering monthly energy consumption in France from 2008 to 2025 \cite{EurostatElectricityConsumption}. The main objective is to identify the most suitable statistical time series models for capturing underlying energy consumption patterns and generating reliable forecasts. Data analysis will include identifying trends, seasonalities, and any anomalies that may influence electricity consumption, providing a deeper understanding of the time series.